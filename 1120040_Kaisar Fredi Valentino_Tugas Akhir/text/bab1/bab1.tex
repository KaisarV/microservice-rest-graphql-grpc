\chapter{PENDAHULUAN}
\section{Latar Belakang}
Perusahaan baru dengan perangkat lunak dengan skalabilitas kecil pada umumnya akan menggunakan arsitektur monolitik, dimana seluruh servis disatukan dalam sebuah server. Hal ini karena perangkat lunak yang dibangun masih relatif kecil dan akan lebih mudah ketika melakukan \textit{deployment}.  Semakin besar ukuran perangkat lunak dengan arsitektur monolitik, maka akan semakin kompleks dan sulit untuk dipelihara dan dikembangkan. Ketika mencari kesalahan dalam kode, diperlukan waktu yang lama karena kompleksitas sistem. Setiap perubahan pada salah satu modul akan memerlukan restart seluruh perangkat lunak, yang dapat memakan waktu lama terutama pada proyek yang besar, hal ini dapat menghambat pengembangan, pengujian, dan pemeliharaan proyek[1]. Permasalahan-permasalahan tersebut dapat diatasi dengan menerapkan arsitektur \textit{microservice}.\\

\textit{Microservice} membuat aplikasi dibangun sebagai sekelompok layanan kecil yang terpisah dan dapat beroperasi secara independen. Kontainerisasi memudahkan pengembangan \textit{microservice} dengan menyediakan wadah terpisah untuk setiap layanan, memberikan manfaat keamanan dan mencegah interaksi antar layanan dalam lingkungan yang sama.  Docker merupakan salah satu platform yang digunakan dalam implementasi konsep ini.\\

\textit{Microservice} harus merupakan komponen yang mandiri secara konseptual, dipasang terpisah satu sama lain, dan memiliki masing-masing memiliki alat penyimpanan data yang terdedikasi seperti \textit{database}[1]. Dengan demikian, masing-masing komponen harus bisa berkomunikasi satu sama lain dengan memberikan \textit{request} dan mengembalikan \textit{response} untuk mengirim dan menerima data. Umumnya, \textit{REST} dianggap sebagai metode standar yang banyak digunakan untuk mengembangkan API[2]. Sebagai alternatif dari \textit{REST}, terdapat teknologi lain seperti \textit{GraphQL}. Meskipun \textit{REST} telah diakui secara luas sebagai metode standar untuk pengembangan \textit{API}, \textit{GraphQL} dianggap sebagai teknologi yang inovatif dalam mengatasi masalah pengambilan data yang menjadi kelemahan utama \textit{REST} yaitu \textit{over-fetching}[2]. Pada tahun 2015 muncul juga teknologi bernama \textit{gRPC} yang dapat menutupi kekurangan \textit{REST} di bidang performa, kecepatan komunikasi, kenyamanan pengembang, dan keamanan dalam komunikasi[3].\\

Sebagai sebuah arsitektur \textit{microservice}, \textit{REST} sudah dikenal sebagai pilihan populer dalam pengembangan \textit{Application Programming Interface (API)}. Namun, munculnya teknologi alternatif yang bernama \textit{GraphQL} dan \textit{gRPC} telah memicu banyak perdebatan dan diskusi tentang keefektifan teknologi mana yang lebih baik dalam melakukan komunikasi antara \textit{client} dan server, terlebih terdapat beberapa skenario yang dapat mempengaruhi kinerja ketiganya[4]. 
Dalam penelitian ini akan dilakukan perbandingan performa antara \textit{REST}, \textit{GraphQL} dan \textit{gRPC}, dan apakah kontainerisasi mempengaruhi kinerja dari ketiga arsitektur tersebut.\\


\section{Rumusan Masalah}
Berdasarkan latar belakang di atas, penelitian ini akan membahas beberapa masalah, sebagai berikut: 
\begin{enumerate}[nolistsep,leftmargin=0.5cm]
  \item Bagaimana pengaruh \textit{under-fetching} terhadap kinerja \textit{REST}, \textit{GraphQL}, dan \textit{gRPC}?
  \item Bagaimana pengaruh \textit{over-fetching} terhadap kinerja kinerja \textit{REST}, \textit{GraphQL}, dan \textit{gRPC}?
  \item Bagaimana \textit{REST} dibandingkan dengan \textit{GraphQL} dan \textit{gRPC} dalam hal \textit{response} time ketika dalam keadaan containerized  dan tidak?
\\
\end{enumerate}

\section{Tujuan Penelitian}
Tujuan penelitian ini adalah untuk memberikan pemahaman yang lebih baik tentang kelebihan dan kelemahan teknologi \textit{REST}, \textit{GraphQL}, dan \textit{gRPC} dalam arsitektur \textit{microservice}, baik dalam keadaan kontainerisasi maupun tidak. Penelitian ini akan menganalisis performa masing-masing teknologi dengan membandingkan kemampuan mereka dalam mengelola permintaan dan respons dari \textit{client}. Diharapkan hasil penelitian ini dapat memberikan informasi terkini dan akurat tentang teknologi mana yang lebih cepat dan efisien dalam mengirim dan menerima data.
% \begin{enumerate}[nolistsep,leftmargin=0.5cm]
%   \item Suspendisse ac porta diam.
%   \item Suspendisse ac porta diam.\\
% \end{enumerate}
\\

\section{Batasan Masalah}
Dalam tugas akhir ini, ada beberapa batasan masalah, antara lain:
\begin{enumerate}[nolistsep,leftmargin=0.5cm]
  \item Bahasa pemrograman yang digunakan adalah \textit{Golang}, untuk mempermudah dan menyederhanakan pembangunan sistem.
  \item Hanya akan ada dua lingkungan yang digunakan untuk menguji perangkat lunak yang dibangun, yaitu \textit{Docker} dan \textit{native operating system (Windows)}.
\\
\end{enumerate}

\section{Konstribusi Penelitian}
Kontribusi yang akan diberikan pada penelitian ini adalah pengujian dilakukan pada lingkungan yang berbeda yaitu \textit{Docker} dan \textit{native operating system (Windows)}.\\

\section{Metodologi Penelitian}
Metode penelitian yang akan dilakukan dalam penelitian ini adalah sebagai berikut:
\begin{enumerate}[nolistsep,leftmargin=0.5cm]
\item Studi Literatur\\
      Pembuatan tugas akhir ini dimulai dengan studi literatur, seperti mengumpulkan paper, jurnal, buku, atau artikel daring mengenai arsitektur \textit{microservice}, \textit{GraphQL}, \textit{REST}, \textit{gRPC} serta kontainerisasi.
\item Analisis Masalah\\
      Dilakukan permasalahan yang ada, batasan-batasan yang dimiliki, dan kebutuhan-kebutuhan yang diperlukan untuk menyelesaikan permasalahan yang ada.
\item Perancangan Sistem dan Arsitektur\\
      Pada tahap ini akan dilakukan perancangan dan pengembangan sistem dengan menerapkan teknologi yang akan diuji yaitu \textit{REST}, \textit{GraphQL} dan \textit{gRPC}. Setelah itu, masing-masing dari sistem yang sudah dibuat akan diletakan pada 2 lingkungan yang berbeda yaitu kontainerisasi dan tidak.
\item Pengujian\\
      Pada tahap ini akan dilakukan pengujian pada sistem yang sudah dirancang. Pengujian akan dilakukan dengan mengukur performa dan penggunaan resource.
\item Evaluasi (Kesimpulan)\\
      Pada tahap ini akan dilakukan pengumpulan dan rangkuman data dari hasil pengujian yang sudah didapatkan kemudian mengambil konklusi dari pengujian yang dilakukan.
\item Dokumentasi\\
      Pada tahap ini akan dilakukan dokumentasi hasil dan analisis dan implementasi secara tertulis dalam bentuk laporan tugas akhir.
\\
\end{enumerate}

\section{Sistematika Pembahasan}
BAB I   PENDAHULUAN\\
Bagian pendahuluan akan memaparkan latar belakang, rumusan masalah, tujuan penelitian, serta batasan masalah yang akan dijelaskan secara terperinci. Selain itu, dijelaskan pula metode penelitian yang akan digunakan untuk menggali data dan hasil penelitian yang akan diperoleh.\\

BAB II	LANDASAN TEORI\\
Bagian landasan teori akan menjelaskan dasar-dasar teori yang menjadi pendukung dari penelitian ini, termasuk di dalamnya arsitektur \textit{microservice}, \textit{REST}, \textit{GraphQL}, \textit{gRPC}, dan Docker.\\

BAB III	METODOLOGI PENELITIAN\\
Metodologi Penelitian menjelaskan tentang prosedur dan langkah-langkah yang digunakan untuk menerapkan teknologi \textit{REST}, \textit{GraphQL}, \textit{gRPC} dan Docker dalam pembuatan program, serta proses pemasangan program pada platform \textit{Docker}.\\

BAB IV	IMPLEMENTASI DAN PENGUJIAN\\
Implementasi dan Pengujian yang berisi pembangunan sistem dan pengujian dengan mensimulasikan dan mengevaluasi program yang telah dibuat.\\

BAB V	KESIMPULAN DAN SARAN\\
Bagian akhir dari penelitian ini berisi kesimpulan dari hasil penelitian yang telah dilakukan dan rekomendasi untuk penelitian selanjutnya di masa depan.\\